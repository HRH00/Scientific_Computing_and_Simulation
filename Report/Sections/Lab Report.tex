\chapter{Lab Report}
This report section will explore the lab work completed throughout the module and discuss skills learned through reflections on each lab. Evidence that each task has been completed is provided.

\section{Lab 1 Naive Fourier Transform}

    The first lab comprised of performing a naive discrete Fourier Transform and Inverse Fourier Transform on a .pgm image file. The Fourier analysis allows for different filters to be applied to the Image. The original image is shown in Figure \autoref{fig:wolf_image}

    \begin{figure}[H]
        \centering
            \includegraphics[width=0.8\columnwidth]{Figures/Week 1/W1-Wolf-Original.png}
            \caption{A screenshot of the original wolf image}
            \label{fig:wolf_image}
    \end{figure}
    
    A Fourier analysis is a mathematical technique which decomposes a waveform into its fundamental sinusoidal sub-components, allowing for the waveform to be represented in the frequency domain, as opposed to the time domain. These Fourier components represent the frequency, amplitude and phase which make up the original signal. By doing this, Fourier analysis allows image processing (among other signal processes) to be performed. The sum of the sub-components is used to recreate the original waveform. 
    

    To begin, supporting Java classes were provided, the are: 
    \begin{itemize}
        \item Display2dFT.java
        \item DisplayDensity.java
        \item ReadPGM.java
        \item SimpleFT  
    \end{itemize}
        
    'DisplayDensity.java' displays a greyscale image rendered from a 2D array of image data, with smaller values displayed as dark and larger values as light.
    
    'ReadPGM.java' parses the .pgm file for use in the skeleton program and 'Display2dFT.java' allows the Fourier Transform output to be represented graphically, using colour to represent imaginary number components. 'SimpleFT' is a skeleton program to code in.
    

    
    \subsection{Discrete Fourier Transform}
    To perform the Discrete Fourier Transform, the equation shown in Figure \autoref{fig:equation-FT} must be implemented, C is a 2D array of length N.
    \begin{center}
        \begin{equation}
            C_{kl} = 1/N^2 \sum_{m=0}^{N-1} \sum_{n=0}^{N-1}\ X_{mn} . e^{-2\pi i(km+nl)/N}
            \label{fig:equation-FT}
        \end{equation}  
    \end{center}%Equation

    The code showing the implementation of the equation is shown in \autoref{fig:wolf-FT-code}
  


    \begin{figure}[H]
        \centering
        \includegraphics[width=1\columnwidth]{Figures/Week 1/W1-SimpleFT-Completed-For-Loop.png}
        \caption{A screenshot of the implementation of the Fourier Transform}
        \label{fig:wolf-FT-code}
    \end{figure}
    
    \begin{figure}[H]
        \centering
        \includegraphics[width=0.8\columnwidth]{Figures/Week 1/W1-FT.png}
        \caption{A screenshot of the Discrete Fourier Transform graphical output}
        \label{fig:wolf-DFT}
    \end{figure}
    
    
    \subsection{Inverse Discrete Fourier Transform}


        \begin{center}
        \begin{equation}
            X_{mn} = \sum_{k=0}^{N-1} \sum_{l=0}^{N-1}\ C_{kl} . e^{2\pi i(km+nl)/N}
            \label{fig:equation-InverseFT}
        \end{equation}  
        \end{center}

        
        \begin{figure}[H]
            \centering
            \includegraphics[width=0.9\columnwidth]{Figures/Week 1/W1-SimpleFT-InverseDFT-Implementation.png}
            \caption{A screenshot of the java inverse DFT implementation}
            \label{fig:inverse-DFT-Code}
    \end{figure}
    
        \begin{figure}[H]
            \centering
            \includegraphics[width=0.8\columnwidth]{Figures/Week 1/W1-SimpleFT-InverseDFT-Graphical-Outputs.png}
            \caption{Graphical output of the program - inverse DFT reconstruction on the right side}
            \label{fig:inverse-DFT-Images-Output}
    \end{figure}
    
        \begin{figure}[H]
            \centering
            \includegraphics[width=1\columnwidth]{Figures/Week 1/W1-SimpleFT-InverseDFT-Test-1.0-code.png}
            \caption{A screenshot of the testing code}
            \label{fig:Testing-Code}
    \end{figure}
    
        \begin{figure}[H]
            \centering
            \includegraphics[width=.49\columnwidth]{Figures/Week 1/W1-SimpleFT-InverseDFT-Test-1.0-Output.png}
            \caption{A screenshot of the testing code output}
            \label{fig:Testing-Code-output}
        \end{figure}
        
        \begin{figure}[H]
            \centering
            \includegraphics[width=1\columnwidth]{Figures/Week 1/W1-SimpleFT-InverseDFT-Floating-Point-Fix.png}
            \caption{A screenshot of code to fix floating point error}
            \label{fig:floating-Point-Fix}
        \end{figure}
        
    
        \begin{figure}[H]
            \centering
            \includegraphics[width=.49\columnwidth]{Figures/Week 1/W1-SimpleFT-InverseDFT-Test-2.0-code.png}
            \caption{A screenshot of code to test the floating point error fix}
            \label{fig:floating-Point-Fixed-test}
        \end{figure}
    
        \begin{figure}[H]
            \centering
            \includegraphics[width=0.8\columnwidth]{Figures/Week 1/W1-SimpleFT-InverseDFT-Test-2.0-CLI-Output.png}
            \caption{A screenshot of console output showing floating point error test has passed}
            \label{fig:floating-Point-Fixed-CLI-Output}
        \end{figure}
        
        \begin{figure}[H]
            \centering
            \includegraphics[width=0.8\columnwidth]{Figures/Week 1/W1-SimpleFT-Completed-For-Loop.png}
            \caption{A screenshot of graphical image reconstruction floating point error fix}
            \label{fig:floating-Point-Fixed-Graphical-Output}
        \end{figure}
    
    
    \subsection{Filtering}
   
    \subsubsection{Low Pass Filter}
          \begin{figure}[H]
        \centering
        \includegraphics[width=0.8\columnwidth]{Figures/Week 1/W1-Low-Pass-Code.png}
        \caption{Screenshot of FT low pass filter code}
        \label{fig:Low-Pass-Filter-code}
      \end{figure}

      \begin{figure}[H]
        \centering
        \includegraphics[width=0.8\columnwidth]{Figures/Week 1/W1-Low-Pass-Truncated.png}
        \caption{Screenshot of the low pass filter truncated FT graphical output}
        \label{fig:Low-Pass-Filter-Truncated}
      \end{figure}

      \begin{figure}[H]
        \centering
        \includegraphics[width=0.8\columnwidth]{Figures/Week 1/W1-Low-Pass-Reconstructed.png}
        \caption{Screenshot of the low pass filter reconstructed image output}
        \label{fig:Low-Pass-Filter-Image}
      \end{figure}
    
    
    \subsubsection{High Pass Filter}
    
      \begin{figure}[H]
        \centering
        \includegraphics[width=1\columnwidth]{Figures/Week 1/W1-High-Pass-Code.png}
        \caption{Screenshot of FT high pass filter code}
        \label{fig:High-Pass-Filter-code}
      \end{figure}

      \begin{figure}[H]
        \centering
        \includegraphics[width=0.8\columnwidth]{Figures/Week 1/W1-High-Pass-Truncated.png}
        \caption{Screenshot of the high pass filter truncated FT graphical output}
        \label{fig:High-Pass-Filter-Truncated}
      \end{figure}

      \begin{figure}[H]
        \centering
        \includegraphics[width=0.8\columnwidth]{Figures/Week 1/W1-High-Pass-Reconstructed.png}
        \caption{Screenshot of the high pass filter reconstructed image output}
        \label{fig:High-Pass-Filter-Image}
      \end{figure}

      \subsubsection{Combined Filters}
        \begin{figure}[H] 
            \centering
            \includegraphics[width=0.8\columnwidth]{Figures/Week 1/W1-Both-Filters.png}
            \caption{Screenshot of the graphical output when filters are combined}
            \label{fig:Combined-Filters}
        \end{figure}



\newpage
\section{Lab 2 - Fast Fourier Transform}
\newpage
This week a Fast Fourier Transform was implemented and benchmarked. Each program was run three times to compare processing speeds. 

\subsection{2D FFT Implementation}
To implement a 2D FFT using the fft1d method from the sample FFT class, a helper function was implemented to transpose the 2D arrays. This is show in \autoref{fig:transpose-code} 

    \begin{figure}[H] 
        \centering
        \includegraphics[width=0.8\columnwidth]{Figures/Week 2/Transpose Implementation.png}
        \caption{Screenshot of Java code showing the transpose function implementation}
        \label{fig:transpose-code}
    \end{figure}

    \begin{figure}[H] 
        \centering
        \includegraphics[width=0.8\columnwidth]{Figures/Week 2/2dFFT Implementation.png}
        \caption{Screenshot of Java code showing the 2D FFT function implementation}
        \label{fig:2dFFT-code}
    \end{figure}


\subsection{Benchmarking}
Benchmarking code was added to the FT and FFT programs to compare run times. Each program from week 1 had its 'System.out.println' removed to allow for more accurate results as the method has a large overhead to call.


    \begin{figure}[H] 
        \centering
        \includegraphics[width=0.8\columnwidth]{Figures/Week 2/Bench code Start.png}
        \caption{Screenshot of Java Benchmarking code at start of main}
        \label{fig:bench-code-start}
    \end{figure}
    
    \begin{figure}[H] 
        \centering
        \includegraphics[width=0.8\columnwidth]{Figures/Week 2/Bench code end.png}
        \caption{Screenshot of Java Benchmarking code at end of main}
        \label{fig:bench-code-end}
    \end{figure}




    \begin{figure}[H] 
        \centering
        \includegraphics[width=0.8\columnwidth]{Figures/Week 2/SimpleFT Bench.png}
        \caption{Screenshot of the Console output from SimpleFT Benchmarking code}
        \label{fig:SimpleFT-Bench}
    \end{figure}
*
    \begin{figure}[H] 
        \centering
        \includegraphics[width=0.8\columnwidth]{Figures/Week 2/SimpleFTFilters Bench.png}
        \caption{Screenshot of the output from SimpleFTFilters Benchmarking code}
        \label{fig:SimpleFTFilters-Bench}
    \end{figure}


    \begin{figure}[H] 
        \centering
        \includegraphics[width=1\columnwidth]{Figures/Week 2/Benchmarking results.png}
        \caption{Screenshot of the bench-marking results table}
        \label{fig:bench-results-table}
    \end{figure}
        

\section{Lab 3 - CT Scanner}

This lab two classes were supplied, DisplaySinogramFT.java outputted the SinogramFT as an image and Sinogram.java contained boilerplate code which includes the initial body slice model. It outputs the initial model and calculated sinogram before outputting a back-projected sinogram. The back-projected image contains very little detail, this is shown on the right side of \autoref{fig:W3-initial-output}.

The initial source model is the 'Shepp Logan Phantom' \textcite{6499235}, which is made of a series of ellipses representing different densities. The class performs an integral Radon Transform on the model densities to produce the Sinogram, the inversion of this transform will produce the final image. 


\begin{figure}[H] 
    \centering
    \includegraphics[width=0.9\columnwidth]{Figures/Week 3/initial-graphics.png}
    \caption{Screenshot of the graphical output of the initial Sinogram.java}
    \label{fig:W3-initial-output}
\end{figure}


\subsection{Applying an FFT}
Filtering the sinogram is required to make details more visible in the back-projected image. In order to perform filtering, a complex FFT is performed over \(\theta\) on the sinogram, the graphical output of this FFT is shown in \autoref{fig:W3-initial-FT}. Implementation code is shown in \autoref{fig:W3-initial-FT-code}.

\begin{figure}[H] 
    \centering
    \includegraphics[width=0.9\columnwidth]{Figures/Week 3/initial-FFTpng.png}
    \caption{Screenshot of the graphical output of the initial Sinogram Fourier Transform}
    \label{fig:W3-initial-FT}
\end{figure}
\begin{figure}[H] 
    \centering
    \includegraphics[width=0.49\columnwidth]{Figures/Week 3/initial-FFT-code.png}
    \caption{Screenshot of the code performing the Sinogram Fourier Transform}
    \label{fig:W3-initial-FT-code}
\end{figure}




\subsection{Applying a Simple Ramp Filter}
Next, the filter must be applied to the Fourier Transform data. A simple Ramp filter has been implemented. A ramp filter is a type of high-pass filter which reduces low frequencies to decrease blurring. The code performing this is shown in \autoref{fig:W3-ramp-code}. For each element, the code generates a $|K|$ value which is then multiplied to both the real and imaginary numerical components of the Fourier Transform.

The value of 'kSigned' is set to 'iK' if the value is $iK <= N/2$, otherwise, the value is set to $iK-N$.   
\begin{figure}[H] 
    \centering
    \includegraphics[width=0.9\columnwidth]{Figures/Week 3/ramp-filter-code.png}
    \caption{Screenshot of the code performing the Ramp filter}
    \label{fig:W3-ramp-code}
\end{figure}

The next step is to rebuild the Sinogram by performing an inverse Fourier Transform over \(\theta\), code is shown in \autoref{fig:W3-inverse-code}.

\begin{figure}[H] 
    \centering
    \includegraphics[width=0.9\columnwidth]{Figures/Week 3/inverse-FT.png}
    \caption{Screenshot of the code performing Inverse FT }
    \label{fig:W3-inverse-code}
\end{figure}

The filtered sinogram is shown in \autoref{fig:W3-filtered-sin}.
\begin{figure}[H] 
    \centering
    \includegraphics[width=0.9\columnwidth]{Figures/Week 3/filtered-sinorgram.png}
    \caption{Screenshot of the graphical output for the filtered sinogram}
    \label{fig:W3-filtered-sin}
\end{figure}

To output the filtered back-projected image, a back-projection operation is performed on the sinogram before the values are normalised. The new image is then drawn. This is shown in \autoref{fig:W3-filtered-back-image}. More detail is visible in the back-projected image, however, a lot of noise has been introduced. To reduce this noise, further filtering is required.
    
\begin{figure}[H] 
    \centering
    \includegraphics[width=1\columnwidth]{Figures/Week 3/filtered-back-projection.png}
    \caption{Screenshot the filtered back-projected sinogram}
    \label{fig:W3-filtered-back-image}
\end{figure}


\subsection{Applying a Ramp Filter with Cutoffs}
The previous filter was altered to make use of a cutoff value, an integer variable 'CUTOFF' was created. This stores the value of: '$N / Value$'. If the '$|K|$' value is greater than this CutOff, the Fourier Component is set to '0'. The code for this implementation is shown in \autoref{fig:W3-filter-with-cutoff-code}.

\begin{figure}[H] 
    \centering
    \includegraphics[width=1\columnwidth]{Figures/Week 3/Filter-with-cuttoff-code.png}
    \caption{Screenshot of the Java code performing filtering with cutoff value}
    \label{fig:W3-filter-with-cutoff-code}
\end{figure}


The program was run with $N/4$ and $N/16$, the back-projected images do have slightly reduced noise. \autoref{fig:W3-many-filters} shows the back-projected images made using a Ramp filter with no CUTOFF (left), with CUTOFF=N/4 (centre) and CUTOFF=N/16 (right).  



\begin{figure}[H] 
    \centering
    \includegraphics[width=1\columnwidth]{Figures/Week 3/filtered-images-ramp-N-4-N-16.png}
    \caption{Screenshot of back project images with filters applied}
    \label{fig:W3-many-filters}
\end{figure}


\newpage
\subsection{Applying Low Pass Cosine Filter}

Next, a low-pass Cosine Filter was applied. The filter takes a similar approach to the previous high-pass filter but applies the following equation. \(|K|  cos(\pi K /(2  *  CUTOFF))\).

Implementation is shown in \autoref{fig:W3-cos-filter-code}. 

\begin{figure}[H] 
    \centering
    \includegraphics[width=1\columnwidth]{Figures/Week 3/filter-cos-codfe.png}
    \caption{Screenshot of the cosine filter code}
    \label{fig:W3-cos-filter-code}
\end{figure}

\autoref{fig:W3-cos-filter-image} shows the image when CUTOFF=$N/4$. The image has little noise removed,
\autoref{fig:W3-cos-filter-image-n-8} shows the image when CUTOFF=$N/8$, the image is now missing a lot of detail compared to the results of the standard ramp filter. 
\begin{figure}[H] 
    \centering
    \includegraphics[width=0.49\columnwidth]{Figures/Week 3/filter-cos-image.png}
    \caption{Screenshot of the cosine filter image Cutoff=N/4}
    \label{fig:W3-cos-filter-image}
\end{figure}
\begin{figure}[H] 
    \centering
    \includegraphics[width=0.49\columnwidth]{Figures/Week 3/filter-cos-image-N-8.png}
    \caption{Screenshot of the cosine filter image Cutoff=N/8}
    \label{fig:W3-cos-filter-image-n-8}
\end{figure}


\newpage
\section{Lab 4 - Radio Interferometry}


\newpage
\section{Lab 5 Lattice gas models}
In this lab, Lattice gas cellular automata were explored in this lab.




\newpage
\section{Lab 6 - Excitable Media} 
\subsection{3 State CA}
This lab consisted of simulating excitable media using cellular automata (CA). To begin, a three-state CA was used to simulate waves. The initial spiral wave output is shown in figure \autoref{fig:W6-initial-3CA}.

\begin{figure}[H] 
    \centering
    \includegraphics[width=0.49\columnwidth]{Figures/Week 6/3stateCA-Initial.png}
    \caption{Screenshot of the graphical output of the initial 3-state CA}
    \label{fig:W6-initial-3CA}
\end{figure}

The Java code was altered to allow for a plane wave simulation. To achieve this, the nested for loop which breaks the initial wave was removed. To allow the wave to repeat itself, the state of all indices on the bottom half of the 'state' matrix was set to '0'. the code and output of the program are shown in Figures \ref{fig:W6-3CA-plane-code} and \ref{fig:W6-3CA-plane-output}. 


\begin{figure}[H] 
    \centering
    \includegraphics[width=0.49\columnwidth]{Figures/Week 6/3stateCA-plane-code.png}
    \caption{Screenshot of the Java code to produce a plane wave}
    \label{fig:W6-3CA-plane-code}
\end{figure}

\begin{figure}[H] 
    \centering
    \includegraphics[width=0.49\columnwidth]{Figures/Week 6/3stateCA-plane-output.png}
    \caption{Screenshot of the graphical output of a plane wave using 3-state CA}
    \label{fig:W6-3CA-plane-output}
\end{figure}



For different simulation starting locations, conditional branching was implemented to allow the user to set the starting location and wave type via a variable in the program. A variable called 'startLocation' must be set to '1', '2' or'3' to define the starting location. \autoref{fig:W6-3CA-conditional-logic-code} shows the code implementing this.

A similar approach was created to allow the user to define the 'doSpiralWave' variable, allowing the wave type to be altered.

\begin{figure}[H] 
    \centering
    \includegraphics[width=0.6\columnwidth]{Figures/Week 6/3stateCA-plane-logic-code.png}
    \caption{Screenshot of the code controlling simulation start-location logic}
    \label{fig:W6-3CA-conditional-logic-code}
\end{figure}

\newpage

\autoref{fig:W6-3CA-start-corner} shows the graphical output when 'startLocation=3', starting the simulation in the corner of the grid. It sets all the 'state' array values to '0', with the exception of a small box of excited cells in one corner.

\begin{figure}[H] 
    \centering
    \includegraphics[width=0.49\columnwidth]{Figures/Week 6/3stateCA-CORNER-output.png}
    \caption{Screenshot of 3-State CA simulation starting in the corner of the grid}
    \label{fig:W6-3CA-start-corner}
\end{figure}

\autoref{fig:W6-3CA-start-middle} shows the graphical output when 'startLocation=2', starting the simulation in the middle of the grid by setting . 
\begin{figure}[H] 
    \centering
    \includegraphics[width=0.49\columnwidth]{Figures/Week 6/3stateCA-MIDDLE-output.png}
    \caption{Screenshot of 3-State CA simulation starting in the middle of the grid}
    \label{fig:W6-3CA-start-middle}
\end{figure}

\subsection{GST}
Next, a Gerhardt-Schuster-Tyson model was implemented to create a spiral and plane wave. The highlighted nested for loop shown in \autoref{fig:W6-GST-plane-code} was removed to create a plane wave simulation. \autoref{fig:W6-GST-side-by-side-output} shows both spiral (left side) and plane wave (right side) simulation outputs.

\begin{figure}[H] 
    \centering
    \includegraphics[width=0.49\columnwidth]{Figures/Week 6/GST-plane-code.png}
    \caption{Screenshot of the GST plane wave code alteration}
    \label{fig:W6-GST-plane-code}
\end{figure}

\begin{figure}[H] 
    \centering
    \includegraphics[width=0.49\columnwidth]{Figures/Week 6/GST-side-by-side-output.png}
    \caption{Screenshot of the graphical output of GST spiral (left) and plane (right) wave simulations}
    \label{fig:W6-GST-side-by-side-output}
\end{figure}


Changing the value of some parameters alters the wave shape, size and stability. For instance, increasing the value of 'K0\_EXCI' to '10' decreases cell excitability. This causes the wave formation to look square, rather than circular, in addition to reducing the width of the wave and slowing the wave propagation speed. This is shown in \autoref{fig:W6-GST-less-excitable}.

\begin{figure}[H] 
    \centering
    \includegraphics[width=0.49\columnwidth]{Figures/Week 6/gst-square.png}
    \caption{Screenshot of the graphical output of GST when cell excitability is reduced}
    \label{fig:W6-GST-less-excitable}
\end{figure}


Altering the 'K0\_RECO' sets the minimum number of unexcited neighbours to set a cell's state to recovery. Increasing the value causes more cells to remain excited after each wave has passed, these cells usually appear near the edges of the matrix, introducing 'noise' into the system. \autoref{fig:W6-GST-high-min-neighbours} shows these pixels at the edges of the grid.


Inversely, reducing this value decreases the number of cells left excited after a wave has passed, seen in \autoref{fig:W6-GST-low-min-neighbours}.
\begin{figure}[H] 
    \centering
    \includegraphics[width=0.49\columnwidth]{Figures/Week 6/GST-high-min-neighbours.png}
    \caption{Screenshot of the graphical output of GST when K0\_RECO is high}
    \label{fig:W6-GST-high-min-neighbours}
\end{figure}

\begin{figure}[H] 
    \centering
    \includegraphics[width=0.49\columnwidth]{Figures/Week 6/GST-low-min-neighbours.png}
    \caption{Screenshot of the graphical output of GST when K0\_RECO is low}
    \label{fig:W6-GST-low-min-neighbours}
\end{figure}

The neighbourhood radius correlates with the width of the wave, increasing this value causes the wave to be thicker \autoref{fig:W6-GST-high-radius}. The origin of the spiral become lower on the grid with higher values.

\begin{figure}[H] 
    \centering
    \includegraphics[width=0.49\columnwidth]{Figures/Week 6/GST-Large-Radius.png}
    \caption{Screenshot of the graphical output of GST when 'R' - radius is increased}
    \label{fig:W6-GST-high-radius}
\end{figure}
\newpage
\subsection{4 State CA}
The initial three-state CA was built upon to add another excited state, state '3' now represents the front of a wave. In addition to the extra state, another array timeToStateChange was created to store the time value until a state may change. \autoref{fig:W6-4CA-dec} shows the variable declaration, \autoref{fig:W6-4CA-init} shows the arrays' value initiation.


\begin{figure}[H] 
    \centering
    \includegraphics[width=1\columnwidth]{Figures/Week 6/4stateCA-time-delaration.png}
    \caption{Screenshot of the 'timeToStateChange' declaration code}
    \label{fig:W6-4CA-dec}
\end{figure}

\begin{figure}[H] 
    \centering
    \includegraphics[width=1\columnwidth]{Figures/Week 6/4stateCA-time-values-init.png}
    \caption{Screenshot of the 'timeToStateChange' initiation code}
    \label{fig:W6-4CA-init}
\end{figure}


The timeToStateChange stores an integer between 0 and 3. In each iteration, all time values are decremented towards 0. This value is updated each state change to the corresponding value in \autoref{tab:state-time-values}.

\begin{table}[htbp]
  \centering
  \renewcommand{\arraystretch}{1.2}
  \begin{tabular}{|c|c|}
    \hline
    \textbf{state} & \textbf{timeToStateChange} \\

    \hline
    1 & 3 \\
    \hline
    2 & 3 \\
    \hline
    3 & 2 \\
    \hline
  \end{tabular}
  \caption{timeToStateChange value when a cell transitions into each state}
  \label{tab:state-time-values}
\end{table}


The program must search for excited neighbours. The neighbour radius has been increased by four to eight neighbours, meaning cells which are diagonally adjacent to the origin cells are included in the search. \autoref{fig:W6-4CA-search-code} Shows that two comparisons are made on each of these cell - setting excitedNeighbour to true, if either a '2' or '3' is found in any of the eight cells. 
\begin{figure}[H] 
    \centering
    \includegraphics[width=1\columnwidth]{Figures/Week 6/4stateCA-neighbour-search-code.png}
    \caption{Screenshot of 'fourStateCA' code searching for excited neighbours}
    \label{fig:W6-4CA-search-code}
\end{figure}

In each iteration, the program must update each 'state' and 'timeToStateChange' value. The first conditional statement, shown in \autoref{fig:W6-4CA-search-code}, decrements the 'timeToStateChange' by one. 

After this the switch statement updates the 'state' and 'timeToStateChange' values dependant on if the cell has excited neighbours or not, and if timeToStateChange is 0.  

\begin{figure}[H] 
    \centering
    \includegraphics[width=1\columnwidth]{Figures/Week 6/4stateCA-update-code.png}
    \caption{Screenshot of fourStateCA Code updating 'state' and 'timeToStateChange' arrays}
    \label{fig:W6-4CA-update-code}
\end{figure}


\autoref{fig:W6-4CA-colour-code} shows an additional colouring condition has been added, allowing a different colour for the new state=3, the colouring scheme is outlined in \autoref{tab:state-color-values}.  

\begin{figure}[H] 
    \centering
    \includegraphics[width=1\columnwidth]{Figures/Week 6/4stateCA-color-code.png}
    \caption{Screenshot of fourStateCA cell colouring code}
    \label{fig:W6-4CA-colour-code}
\end{figure}

\vspace{10mm}
\begin{table}[htbp]
  \centering
  \begin{tabular}{|c|c|c|}
    \hline
    \textbf{State Number} &\textbf{State Meaning} & \textbf{Cell Colour} \\

    \hline
    0 & Resting & Background (White)\\
    \hline
    1 & Recovering & Blue \\
    \hline
    2 & Excited-Wave-Plateau & Dark Red \\
    \hline
    3 & Excited-Wave-Front & Bright Red \\
    \hline
  \end{tabular}
  \caption{'fourStateCA' Cell Colour Scheme}
  \label{tab:state-color-values}
\end{table}

The graphical output for the program is shown in \autoref{fig:W6-4CA-output}.
\begin{figure}[H] 
    \centering
    \includegraphics[width=0.49\columnwidth]{Figures/Week 6/4stateCA-output.png}
    \caption{Screenshot of 'fourStateCA' graphical output}
    \label{fig:W6-4CA-output}
\end{figure}
\newpage
\section{Lab 7}
\newpage
\section{Lab 8}
\newpage
\section{Lab 10}
